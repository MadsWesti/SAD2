\documentclass[a4paper,11pt]{article}
\usepackage[utf8]{inputenc}
\usepackage{listings}
\usepackage{mathtools}   % need for subequations
\usepackage{graphicx}   % need for figures
\usepackage{verbatim}   % useful for program listings
\usepackage{color}      % use if color is used in text
\usepackage{hyperref}  
\usepackage{url}
\usepackage{float}
\usepackage{todonotes}
\usepackage{tikz}
\usepackage{enumitem}
\usepackage{hyperref}
\usepackage{pdfpages}
\usepackage{caption}
\usepackage{subcaption}
\usepackage{listings}
\usepackage{color}
\usepackage{amsfonts}
\usepackage{latexsym}
\usepackage[T1]{fontenc} % use for allowing < and > in cleartext
\usepackage{fixltx2e}    % use for textsubscript

\usepackage[margin=2.5cm]{geometry}
\usepackage[ampersand]{easylist}
\usepackage{tikz}
\usepackage{epstopdf}

\bibliographystyle{plain}

\begin{document}
\setlength{\parindent}{0cm}
\setlength{\unitlength}{1mm}
\date{December 17th 2014\\ IT University of Copenhagen}
\title{Roles In Movies\\SAD2 Fall 2014}

\author{Sarah de Voss\\
\texttt{satv@itu.dk}
\and Elvis Flesborg\\
\texttt{efle@itu.dk}
\and Mads Westi\\
\texttt{mwek@itu.dk}}
\clearpage\maketitle
\newpage
\thispagestyle{empty}
\setcounter{page}{1}
\tableofcontents
\newpage

\section{The method acting problem}
Acting is hard, the specific branch of method acting\footnote{Actors use this technique to create in themselves the thoughts and feelings of their role} appear even harder. Whenever an actor has finished playing a role, the process of creating a character starts all over. This is time consuming and inefficient, but what if the actor could use parts of his or hers current role? We want to construct an algorithm that gives actors clues to which roles to audition for next, by measuring the similarity between roles. We have dubbed this similarity search problem the "Method acting problem".


\section{Implementation}
We assume that there is a high correlation between the distinct roles in a movie and the rating the movie receives. Because the success of an actor is dependent on, whether he or she appears in high ranked movies, therefore movies with $rank \leq 7.0$ is purged from the data.

We will look at the naïve solution, then \emph{MinHashing}, and then \emph{Locality Sensitive Hashing}, this is part one of the report. After that we will implement \emph{b-bit} and \emph{odd sketch}, which is the extension to our report.


\subsection{Data representation}
A role is represented as a list of movies\footnote{Corresponding to the ids of the movies, which is integers} the specific role appear in.

\begin{equation}
r_1 = \{m_1, m_2, m_3, \ldots , m_i\}
\end{equation}

\subsection{Measuring similarity}
Since a role is represented as a set movies, we can measure the similarity between roles using the Jaccard similarity, where the similarity between two sets $S_1$ and $S_2$.

\begin{equation}
Sim(S_1, S_2) = \frac{|S_1 \cap S_2|}{|S_1 \cup S_2|}
\end{equation}
The Jaccard index falls in the range of 0-1. It is 1 if the sets are similar, and it is 0 if the sets are disjoint.


\subsection{Measuring Running Time}
For every new implementation, we will measure the actual running time. It will be measured on a normal laptop computer with 8GB RAM and and intel i5 quadcore. Though, if the running time is taking longer than out patience permits, we will just put in how long we waited, and note that we did not run it through.

We will also create some random dummy data to run. This way we will be able to graph the running times, and look at the tendency curves. In the dummy data, there are 10 times as many roles, as there are movies. This corresponds pretty well with the \emph{imdb-r.txt} file, which has 57,736 distinct movies, and 553,263 distinct roles (not counting the movies without any roles and roles with nonexisting movie-ids).

But even though there are 10 more roles per movies. Many of these roles are only in one movie, and only few roles are in many movies. If we calculate the average number of movies, that a role appears in, it is somewhere between 1.2 and 2.0, depending on what our cut-off rank is (see figure ~\ref{fig:sparsenesscutoff}. We have defined our minimum rank cutoff to movies with at least 7.0 in rank, which corresponds to an average of 1.7 movies per role. This has influence on the characteristic matrix, that we create when parsing.

This will be taken into account when creating our random dummy data.

\begin{figure}
    \begin{center}
        \input{plots/ones_per_role.tex}
        \caption{Sparseness with cutoff}
        \label{fig:sparsenesscutoff}
    \end{center}
\end{figure}

\subsection{Naïve Implementation}

The data is stored in a characteristic matrix, where the rows are movies and the columns are roles.
\begin{equation}
movies \times roles = 
\begin{bmatrix}
    1 & 0 & 1 & 0\\
    1 & 0 & 0 & 1\\
    0 & 1 & 0 & 1\\
    0 & 1 & 0 & 1\\
    0 & 1 & 0 & 1\\
    1 & 0 & 1 & 0\\
    1 & 0 & 1 & 0
\end{bmatrix}
\end{equation}

Because we have to permute every possible pair of roles, the running time is $O(n^2 \cdot m)$, where $n$ is the number of distinct roles, and $m$ is the number of movies. In a scenario where $m$ is almost equal to $n$, the runtime will be $O(n^3)$. \\

This will surely not fit in memory with the whole dataset, and more sofisticated algorithms are needed.

\subsubsection{Actual Running Time}
The actual running time with the naïve implementation takes too long for our normal laptop computer to outrun our patience. We ran the implementation on random dummy datasets, which can be seen in figure ~\ref{fig:naive_at}.

\begin{figure}
    \begin{center}
        \input{plots/naive_at.tex}
        \caption{Naïve Implementation Actual Times}
        \label{fig:naive_at}
    \end{center}
\end{figure}


\section{Min Hashing}
Since the permutation of such a large matrix is infeasible, we create a number of hash functions, $k$, which should be less than the number of rows in the characteristic matrix. 

Remarkably, the probability that the minhash function produces the same value for two sets is the same as the Jaccard similarity of the two sets (p. 82 in ch3). Let $x$ be the rows in which.


The signatures are created in the following way: For every row in the characteristic matrix run all the hash functions for the row index. When this is done, we have created the signature matrix, which is a $k\times n$ matrix.\\

We are using universal hashing\footnote{$(ax + b) \bmod p$} as our hashing functions, where there will be some collisions in the hashes. Some row indexes will be hashed to the same, but as long as $k$ is large it is not important (p. 83-84 in ch3). \\

We still have to permute all pairs of roles, but the number of comparisons is reduced to $k$, the new running time is therefore $O(n^2 \cdot k)$.

\subsection{Actual Running Times}
In figure ~\ref{fig:minhashing_at} the actual running times can be seen. Compared to the running times of the naïve solution, it is a lot faster.


\begin{figure}
    \begin{center}
        \input{plots/minhashing_at.tex}
        \caption{MinHashing Actual Times}
        \label{fig:minhashing_at}
    \end{center}
\end{figure}


\section{Locality Sensitive Hashing (LSH)}
Locality sensitive hashing is a way of coping with an even larger amount of data than MinHashing can. So if the number of pairs is too large LSH is an option to use. The idea is that similar items hash to the same bucket and then we only check the similarity of items in the same bucket. From the MinHash algorithm we have MinHash signatures. For each hashing we take the union of the sets of each hash function.

\subsection{How does it work?}
We take the signatures produced by min hashing and chop them up unto shorter vectors by dividing the total number of rows int $b$ bands of $r$ rows. Total number of rows = $r\times b$ We then hash the vectors into a large number of buckets. If 2 vectors hash to the same bucket there is a good chance that they are similar. These vectors become candidate pairs. This leaves us with a significant fraction of the original set of pairs to check with jaccard similarity.\\

When choosing the right number of bands $b$ which then gives us the number of rows $r$ it is important to keeping in mind that we want few false negatives/positives but we also want speed. We can compute the probability of a particular pair becoming a candidate pair by using the S-curve equation. $s$ is the probability that 2 signatures become a candidate pair.

\begin{equation}
\text {S-curve equation} = 1 - (1 - s^r)^b 
\end{equation}\\

How much accuracy do we sacrifice in percent?\\

Run time analysis/space complexity
While the use of minhashing reduces the running time of the similarity calculations. It is still not fesible to use the algorithm on the IMDB data set. The general idea of LSH is that before making comparisions, the signtures are hashed in such a way that similar signatures are hashed to the same buckets, resulting in candidate pairs for comparision.\\

If you were an algorithm which algorithm would you be?

\section{B-bit}
B.bit minwise hashing is a way to estimate set similarity in a storage efficient way.
\section{Odd sketch}

\section{Conclusion}
\end{document}





